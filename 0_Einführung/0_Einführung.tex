% Options for packages loaded elsewhere
\PassOptionsToPackage{unicode}{hyperref}
\PassOptionsToPackage{hyphens}{url}
%
\documentclass[
  8pt,
  ignorenonframetext,
]{beamer}
\usepackage{pgfpages}
\setbeamertemplate{caption}[numbered]
\setbeamertemplate{caption label separator}{: }
\setbeamercolor{caption name}{fg=normal text.fg}
\beamertemplatenavigationsymbolsempty
% Prevent slide breaks in the middle of a paragraph
\widowpenalties 1 10000
\raggedbottom
\setbeamertemplate{part page}{
  \centering
  \begin{beamercolorbox}[sep=16pt,center]{part title}
    \usebeamerfont{part title}\insertpart\par
  \end{beamercolorbox}
}
\setbeamertemplate{section page}{
  \centering
  \begin{beamercolorbox}[sep=12pt,center]{part title}
    \usebeamerfont{section title}\insertsection\par
  \end{beamercolorbox}
}
\setbeamertemplate{subsection page}{
  \centering
  \begin{beamercolorbox}[sep=8pt,center]{part title}
    \usebeamerfont{subsection title}\insertsubsection\par
  \end{beamercolorbox}
}
\AtBeginPart{
  \frame{\partpage}
}
\AtBeginSection{
  \ifbibliography
  \else
    \frame{\sectionpage}
  \fi
}
\AtBeginSubsection{
  \frame{\subsectionpage}
}
\usepackage{amsmath,amssymb}
\usepackage{lmodern}
\usepackage{iftex}
\ifPDFTeX
  \usepackage[T1]{fontenc}
  \usepackage[utf8]{inputenc}
  \usepackage{textcomp} % provide euro and other symbols
\else % if luatex or xetex
  \usepackage{unicode-math}
  \defaultfontfeatures{Scale=MatchLowercase}
  \defaultfontfeatures[\rmfamily]{Ligatures=TeX,Scale=1}
\fi
% Use upquote if available, for straight quotes in verbatim environments
\IfFileExists{upquote.sty}{\usepackage{upquote}}{}
\IfFileExists{microtype.sty}{% use microtype if available
  \usepackage[]{microtype}
  \UseMicrotypeSet[protrusion]{basicmath} % disable protrusion for tt fonts
}{}
\makeatletter
\@ifundefined{KOMAClassName}{% if non-KOMA class
  \IfFileExists{parskip.sty}{%
    \usepackage{parskip}
  }{% else
    \setlength{\parindent}{0pt}
    \setlength{\parskip}{6pt plus 2pt minus 1pt}}
}{% if KOMA class
  \KOMAoptions{parskip=half}}
\makeatother
\usepackage{xcolor}
\newif\ifbibliography
\setlength{\emergencystretch}{3em} % prevent overfull lines
\providecommand{\tightlist}{%
  \setlength{\itemsep}{0pt}\setlength{\parskip}{0pt}}
\setcounter{secnumdepth}{-\maxdimen} % remove section numbering
% type setting
% ------------------------------------------------------------------------------
\usepackage[german]{babel}     

% fonts
% ------------------------------------------------------------------------------
\usefonttheme{professionalfonts}

% slide title and horizontal line
% ------------------------------------------------------------------------------
\setbeamertemplate{frametitle}{%
    \vskip-30pt \color{black}\large%
    \begin{minipage}[b][23pt]{120mm}%
    \flushleft\insertframetitle%
    \end{minipage}%
}

\setbeamertemplate{headline}										
{
\vskip10pt\hfill\hspace{3.5mm} 										 
\vskip15pt\color{black}\rule{\textwidth}{0.4pt} 					 
}

% slide number
% ---------------------------------------------------------------
\setbeamertemplate{navigation symbols}{}
\setbeamertemplate{footline}
{
\vskip5pt
\vskip2pt
\makebox[123mm]{\hspace{7.5mm}
\hfill Allgemeines Lineares Modell - Tutorium $\vert$ 
Sean Mulready $\vert$
Folie \insertframenumber}
\vskip4pt
}

% block color scheme
% ------------------------------------------------------------------------------
% colors
\definecolor{white}{RGB}{255,255,255}
\definecolor{grey}{RGB}{205,205,205}
\definecolor{lightgrey}{RGB}{245,245,245}
\definecolor{LightBlue}{RGB}{220,220,255}
\definecolor{darkblue}{RGB}{51, 51, 153}
\definecolor{darkcyan}{RGB}{0,102,102}
\definecolor{middlecyan}{RGB}{0,153,153}
\definecolor{darkgreen}{RGB}{0,102,51}
\definecolor{plum}{RGB}{128,0,128}
\definecolor{orange}{RGB}{255,141,42}

% definitions and theorems
\setbeamercolor{block title}{fg = black, bg = grey}
\setbeamercolor{block body}{fg = black, bg = lightgrey}

% general line spacing 
% ------------------------------------------------------------------------------
\linespread{1.3}

% local line spacing
% ------------------------------------------------------------------------------
\usepackage{setspace}

% colors
% -----------------------------------------------------------------------------
\usepackage{color}

% justified text
% ------------------------------------------------------------------------------
\usepackage{ragged2e}
\usepackage{etoolbox}
\apptocmd{\frame}{}{\justifying}{}

% bullet point lists
% -----------------------------------------------------------------------------
\setbeamertemplate{itemize item}[circle]
\setbeamertemplate{itemize subitem}[circle]
\setbeamertemplate{itemize subsubitem}[circle]
\setbeamercolor{itemize item}{fg = black}
\setbeamercolor{itemize subitem}{fg = black}
\setbeamercolor{itemize subsubitem}{fg = black}
\setbeamercolor{enumerate item}{fg = black}
\setbeamercolor{enumerate subitem}{fg = black}
\setbeamercolor{enumerate subsubitem}{fg = black}
\setbeamerfont{itemize/enumerate body}{}
\setbeamerfont{itemize/enumerate subbody}{size = \normalsize}
\setbeamerfont{itemize/enumerate subsubbody}{size = \normalsize}

% color links
% ------------------------------------------------------------------------------
\usepackage{hyperref}
\definecolor{urls}{RGB}{204,0,0}
\hypersetup{colorlinks, citecolor = darkblue, urlcolor = urls}


% additional math commands
% ------------------------------------------------------------------------------
\usepackage{bm}
\usepackage{mathtools}
\usepackage{graphics}
\usepackage{amsmath, amssymb}                                         
\newcommand{\niton}{\not\owns}
\DeclareMathOperator*{\intinf}{\int_{-\infty}^{\infty}}
\DeclareSymbolFont{extraitalic}      {U}{zavm}{m}{it}
\DeclareMathSymbol{\Qoppa}{\mathord}{extraitalic}{161}
\DeclareMathSymbol{\qoppa}{\mathord}{extraitalic}{162}



% text highlighting
% ------------------------------------------------------------------------------
\usepackage{soul}
\makeatletter
\let\HL\hl
\renewcommand\hl{%
  \let\set@color\beamerorig@set@color
  \let\reset@color\beamerorig@reset@color
  \HL}
\makeatother

% equation highlighting
% -----------------------------------------------------------------------------
\newcommand{\highlight}[2][yellow]{\mathchoice%
  {\colorbox{#1}{$\displaystyle#2$}}%
  {\colorbox{#1}{$\textstyle#2$}}%
  {\colorbox{#1}{$\scriptstyle#2$}}%
  {\colorbox{#1}{$\scriptscriptstyle#2$}}}%

% additional mathematical operators
% ------------------------------------------------------------------------------
\DeclareMathOperator*{\argmax}{arg\,max}
\DeclareMathOperator*{\argmin}{arg\,min}
\usepackage{xfrac}
\usepackage{units}


\ifLuaTeX
  \usepackage{selnolig}  % disable illegal ligatures
\fi
\IfFileExists{bookmark.sty}{\usepackage{bookmark}}{\usepackage{hyperref}}
\IfFileExists{xurl.sty}{\usepackage{xurl}}{} % add URL line breaks if available
\urlstyle{same} % disable monospaced font for URLs
\hypersetup{
  hidelinks,
  pdfcreator={LaTeX via pandoc}}

\author{}
\date{\vspace{-2.5em}}

\begin{document}

\begin{frame}[plain]{}
\protect\hypertarget{section}{}
\center

\begin{center}\includegraphics[width=0.2\linewidth]{../Abbildungen/wtfi_otto} \end{center}

\vspace{2mm}

\huge

Tutorium

\Large

Allgemeines Lineares Modell \vspace{4mm}

\normalsize

BSc Psychologie SoSe 2023

\vspace{12mm}
\normalsize

Sean Mulready

\vspace{3mm}
\scriptsize

Inhalte basieren auf Kursmaterialien für
\href{https://www.ipsy.ovgu.de/Institut/Abteilungen+des+Institutes/Methodenlehre+I+_+Experimentelle+und+Neurowissenschaftliche+Psychologie/Lehre/Sommersemester+2022/Allgemeines+Lineares+Modell.html}{\textcolor{darkblue}{ALM Adresse noch ändern}}
von
\href{https://www.ipsy.ovgu.de/Institut/Abteilungen+des+Institutes/Methodenlehre+I+_+Experimentelle+und+Neurowissenschaftliche+Psychologie/Team/Dirk+Ostwald.html}{Dirk
Ostwald}, lizenziert unter
\href{https://creativecommons.org/licenses/by-sa/4.0/deed.de}{CC
BY-NC-SA 4.0}
\end{frame}

\begin{frame}[plain]{}
\protect\hypertarget{section-1}{}
\huge
\center
\vfill

Hallo! :) \vfill
\end{frame}

\begin{frame}[plain]{}
\protect\hypertarget{section-2}{}
\huge
\center
\vfill

(0.1) Materialien, Links, Organisatorisches \vfill
\end{frame}

\begin{frame}{}
\protect\hypertarget{section-3}{}
\href{https://www.ipsy.ovgu.de/methodenlehre_I-path-980,1404.html}{\textcolor{darkblue}{Homepage}}
\vspace{3mm}

\begin{center}\includegraphics[width=0.75\linewidth]{../Abbildungen/Lehrstuhlseite} \end{center}
\end{frame}

\begin{frame}{Organisatorisches}
\protect\hypertarget{organisatorisches}{}
\vspace{3mm}
\setstretch{1}

Termine:

\begin{itemize}
\tightlist
\item
  jeden Mittwoch

  \begin{itemize}
  \tightlist
  \item
    Gruppe 1: 09.15 Uhr - 10.45 Uhr {[}G22A R120{]}
  \item
    Gruppe 2: 13.15 Uhr - 14.45 Uhr {[}G22A R120{]}
  \item
    evtl. Abklärung zwecks 24.05. (Termin nach Christi Himmelfahrt)
  \end{itemize}
\end{itemize}
\end{frame}

\begin{frame}{Materialien etc.}
\protect\hypertarget{materialien-etc.}{}
\setstretch{2}

\begin{itemize}
\tightlist
\item
  Folien \& Videos (Vorlesung \& Tutorium) gibt es auf der
  \href{https://www.ipsy.ovgu.de/Institut/Abteilungen+des+Institutes/Methodenlehre+I+_+Experimentelle+und+Neurowissenschaftliche+Psychologie/Lehre/Sommersemester+2023/Allgemeines+Lineares+Modell.html}{\textcolor{darkblue}{Kurswebsite}}
\item
  Q\&A (per dm oder im Forum) via
  \href{https://mm.cs.ovgu.de/bsc-psy-2022/channels/b2-alm}{\textcolor{darkblue}{Mattermost}}

  \begin{itemize}
  \tightlist
  \item
    Einmalige Registrierung zum Team ``bsc-psy-2022'' über diesen
    \href{https://mm.cs.ovgu.de/signup_user_complete/?id=6zhrsn3oab8pdynz16gmqpm9ka}{\textcolor{darkblue}{Link}}
  \end{itemize}
\item
  Code auf \href{Link\%20zu\%20Github}{\textcolor{darkblue}{Github}}
\item
  Ankündigungen über
  \href{https://elearning.ovgu.de/course/view.php?id=14470}{\textcolor{darkblue}{Moodle}}
\item
  Die vorherige Iteration des Kurses gibt es
  \href{https://www.ipsy.ovgu.de/Institut/Abteilungen+des+Institutes/Methodenlehre+I+_+Experimentelle+und+Neurowissenschaftliche+Psychologie/Lehre/Sommersemester+2022/Allgemeines+Lineares+Modell.html}{\textcolor{darkblue}{hier}}
\end{itemize}
\end{frame}

\begin{frame}{Ziele des Tutoriums}
\protect\hypertarget{ziele-des-tutoriums}{}
\setstretch{2}

\begin{itemize}
\tightlist
\item
  Wiederholung der Kerninhalte der Vorlesung
\item
  Beantwortung der Selbstkontrollfragen
\item
  Klärung aller offenen Fragen
\end{itemize}
\end{frame}

\begin{frame}{Tipps}
\protect\hypertarget{tipps}{}
\setstretch{2}

\begin{itemize}
\tightlist
\item
  Auswendig lernen ist ein guter Anfang für das Verstehen
\item
  Versuchen, zu verstehen - je mehr ich verstehe desto weniger muss ich
  auswendig behalten
\item
  Formeln von Hand abschreiben oder in einem Satz formulieren kann auch
  helfen
\item
  Manches wird erst im Nachhinein so richtig klar
\end{itemize}
\end{frame}

\begin{frame}[plain]{}
\protect\hypertarget{section-4}{}
\huge
\center
\vfill

Noch Fragen? \vfill
\end{frame}

\begin{frame}{Motivation}
\protect\hypertarget{motivation}{}
\begin{center}\includegraphics[width=0.45\linewidth]{../Abbildungen/mot_2} \includegraphics[width=0.45\linewidth]{../Abbildungen/mot_1} \end{center}
\end{frame}

\end{document}
